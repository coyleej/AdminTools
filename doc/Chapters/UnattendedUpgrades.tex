Ubuntu by default will automatically download and install security updates but not security upgrades. We will use \texttt{unattended-upgrades} to automate the latter. It's preinstalled in 18.04 and can be configured in a single command. 

Run \texttt{sudo dpkg-reconfigure unattended-upgrades} and follow the prompts. The whole process takes less than a minute. Detailed documentation can be found \href{https://help.ubuntu.com/lts/serverguide/automatic-updates.html}{here}.

%%%% 
%Removed the manual setup from my script. Updates to the unattended-upgrades package made some of this irrelevant, and I never got the automated setup properly working in the first place.
%%%%
%\item Configure it manually by editing \texttt{/etc/apt/apt.conf.d/50unattended-upgrades}. The setup for our servers is described below. Additional details can be found at these locations:
%\href{https://libre-software.net/ubuntu-automatic-updates/}{1}, 
%\href{https://www.howtoforge.com/tutorial/how-to-setup-automatic-security-updates-on-ubuntu-1604/}{2}, 
%\href{https://linoxide.com/ubuntu-how-to/enable-disable-unattended-upgrades-ubuntu-16-04/}{3}, and 
%\href{https://www.linuxbabe.com/ubuntu/automatic-security-update-unattended-upgrades-ubuntu-18-04}{4}. 
%%1: Best step-by-step instructions 
%%2: Good overview
%%3:Ok guide (skip SMTP relay section)
%
%	\begin{enumerate}
%	\item Install \texttt{bsd-mailx}. In the installation process, pick ``local only" and use ``<hostname>" as the mail server host.
%
%	\item Navigate to \texttt{/etc/apt/apt.conf.d/}.
%
%	\item Copy \texttt{50unattended-upgrades} to \texttt{50unattended-upgrades.bak}. Open \texttt{50unattended-upgrades} and change the following:
%		\begin{enumerate}
%		\item Uncomment \texttt{Unattended-Upgrade::Mail} and change the address to \texttt{<admin\_account>}. 
%
%		\item Set it to send you mail only when there are errors. (The default is sending mail every time it updates.) Messages are placed in \texttt{/var/mail}. Uncomment the following: \\
%		\texttt{Unattended-Upgrade::MailOnlyOnError "true"}
%			
%		\item All other items are left as the defaults.
%		\end{enumerate}
%
%	\item Copy \texttt{20auto-upgrades} to \texttt{20auto-upgrades.bak}. Open \texttt{20auto-upgrades} and make sure that it contains the following (time intervals are in days):
%		\begin{verbatim}
%		APT::Periodic::Update-Package-Lists "1";
%		APT::Periodic::Download-Upgradeable-Packages "1";
%		APT::Periodic::AutocleanInterval "7";
%		APT::Periodic::Unattended-Upgrade "1";
%		\end{verbatim}
%
%	\item Test: \texttt{sudo unattended-upgrades --dry-run --debug}
%
%	\item If the dry run worked: \texttt{sudo rm *.bak}
%	\end{enumerate}
%
%\end{itemize}

If things ever go wrong, you may need to check the log files:

	\texttt{/var/log/unattended-upgrades/unattended-upgrades.log}
	
	\texttt{/var/log/apt}

By default \texttt{unattended-upgrades} runs randomly within a twelve hour block to smooth out demand on the mirrors. This is fine for our purposes and does not need modification.


%	\begin{enumerate}
%	\item To view the default values: 
%
%	\texttt{cat /lib/systemd/system/apt-daily.timer}
%	
%	\item If you wish to change the values, do \textbf{NOT} modify the default file. Instead, you should override it. Create a new file:
%
%	\texttt{/etc/systemd/system/apt-daily-upgrade.timer.d/override.conf}
%
%	Add something like the following (run once a day at 6:00, randomized delay of $<$4 hours):
%	\begin{verbatim}
%[Timer]
%OnCalendar=*-*-* 6:00
%RandomizedDelaySec=4h
%	\end{verbatim}
%
%	\item Reload and restart the daemons:
%	\begin{verbatim}
%sudo systemctl daemon-reload && sudo systemctl restart apt-daily.timer
%	\end{verbatim}
%
%	\end{enumerate}

