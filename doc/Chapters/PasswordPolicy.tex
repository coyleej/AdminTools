Passwords must be at least 15 characters long, with at least two upper case letters, two lower case letters, two numbers, and two special characters. They must expire after 60 days and contain at least two characters not in the previous password. Running \texttt{set\_passwd\_policy.sh} will automatically change these settings.

\begin{enumerate}
\item Change the password expiration settings. Open \texttt{/etc/login.defs} and set these variables:

	\begin{verbatim}
	PASS_MAX_DAYS   60
	PASS_WARN_AGE   7
	\end{verbatim}

\item Set the password requirements. Open \texttt{/etc/security/pwquality.conf}. Negative values indicate that that number of the thing be present in a new password.

	\begin{verbatim}
	difok = 2
	minlen = 15
	dcredit = -2
	ucredit = -2
	lcredit = -2
	ocredit = -2
	minclass = 4
	maxrepeat = 2
	usercheck = 1
	\end{verbatim}

\item Changes to \texttt{/etc/login.defs} only affect new users \href{https://www.server-world.info/en/note?os=Ubuntu_16.04&p=password}{(source)}. You must also apply these changes to existing users: \texttt{sudo chage -M <days> <user>}

	The following code automates these changes (you can confirm changes with \texttt{sudo chage -l <user>}):

	\begin{verbatim}
	userlist=$(grep "10[0-9][0-9]" /etc/passwd | cut -d ":" -f 1)
	for user in $userlist; do
	    sudo chage -M 60 $user
	    sudo chage -l $user | grep "Pass.*expire"
	done
	\end{verbatim}

\end{enumerate}
